\documentclass[journal,12pt,twocolumn]{IEEEtran}
%
\usepackage{setspace}
\usepackage{gensymb}
\usepackage{xcolor}
\usepackage{caption}
%\usepackage{subcaption}
%\doublespacing
\singlespacing
\usepackage{multicol}

\usepackage{iithtlc}
%\usepackage{graphicx}
%\usepackage{amssymb}
%\usepackage{relsize}
\usepackage[cmex10]{amsmath}
\usepackage{mathtools}
%\usepackage{amsthm}
%\interdisplaylinepenalty=2500
%\savesymbol{iint}
%\usepackage{txfonts}
%\restoresymbol{TXF}{iint}
%\usepackage{wasysym}
\usepackage{amsthm}
\usepackage{mathrsfs}
\usepackage{txfonts}
\usepackage{stfloats}
\usepackage{cite}
\usepackage{cases}
\usepackage{subfig}
%\usepackage{xtab}
\usepackage{longtable}
\usepackage{multirow}
%\usepackage{algorithm}
%\usepackage{algpseudocode}
\usepackage{enumitem}
\usepackage{mathtools}
%\usepackage{stmaryrd}

\usepackage{listings}
    \usepackage[latin1]{inputenc}                                 %%
    \usepackage{color}                                            %%
    \usepackage{array}                                            %%
    \usepackage{longtable}                                        %%
    \usepackage{calc}                                             %%
    \usepackage{multirow}                                         %%
    \usepackage{hhline}                                           %%
    \usepackage{ifthen}                                           %%
  %optionally (for landscape tables embedded in another document): %%
    \usepackage{lscape}     
\usepackage{circuitikz}
%\usepackage{wasysym}
%\newcounter{MYtempeqncnt}
\DeclareMathOperator*{\Res}{Res}
%\renewcommand{\baselinestretch}{2}
\renewcommand\thesection{\arabic{section}}
\renewcommand\thesubsection{\thesection.\arabic{subsection}}
\renewcommand\thesubsubsection{\thesubsection.\arabic{subsubsection}}

\renewcommand\thesectiondis{\arabic{section}}
\renewcommand\thesubsectiondis{\thesectiondis.\arabic{subsection}}
\renewcommand\thesubsubsectiondis{\thesubsectiondis.\arabic{subsubsection}}

% correct bad hyphenation here
\hyphenation{op-tical net-works semi-conduc-tor}

\def\inputGnumericTable{}  

\lstset{
language=python,
frame=single, 
breaklines=true
}

\begin{document}
%

\theoremstyle{definition}

\newtheorem{theorem}{Theorem}[section]
\newtheorem{problem}{Problem}
\newtheorem{proposition}{Proposition}[section]
\newtheorem{lemma}{Lemma}[section]
\newtheorem{corollary}[theorem]{Corollary}
\newtheorem{example}{Example}[section]
\newtheorem{definition}{Definition}[section]
%\newtheorem{algorithm}{Algorithm}[section]
%\newtheorem{cor}{Corollary}
\newcommand{\BEQA}{\begin{eqnarray}}
\newcommand{\EEQA}{\end{eqnarray}}
\newcommand{\define}{\stackrel{\triangle}{=}}

\bibliographystyle{IEEEtran}
%\bibliographystyle{ieeetr}



\providecommand{\pr}[1]{\ensuremath{\Pr\left(#1\right)}}
\providecommand{\qfunc}[1]{\ensuremath{Q\left(#1\right)}}
\providecommand{\sbrak}[1]{\ensuremath{{}\left[#1\right]}}
\providecommand{\lsbrak}[1]{\ensuremath{{}\left[#1\right.}}
\providecommand{\rsbrak}[1]{\ensuremath{{}\left.#1\right]}}
\providecommand{\brak}[1]{\ensuremath{\left(#1\right)}}
\providecommand{\lbrak}[1]{\ensuremath{\left(#1\right.}}
\providecommand{\rbrak}[1]{\ensuremath{\left.#1\right)}}
\providecommand{\cbrak}[1]{\ensuremath{\left\{#1\right\}}}
\providecommand{\lcbrak}[1]{\ensuremath{\left\{#1\right.}}
\providecommand{\rcbrak}[1]{\ensuremath{\left.#1\right\}}}
\theoremstyle{remark}
\newtheorem{rem}{Remark}
\newcommand{\sgn}{\mathop{\mathrm{sgn}}}
\providecommand{\abs}[1]{\left\vert#1\right\vert}
\providecommand{\res}[1]{\Res\displaylimits_{#1}} 
\providecommand{\norm}[1]{\lVert#1\rVert}
\providecommand{\mtx}[1]{\mathbf{#1}}
\providecommand{\mean}[1]{E\left[ #1 \right]}
\providecommand{\fourier}{\overset{\mathcal{F}}{ \rightleftharpoons}}
%\providecommand{\hilbert}{\overset{\mathcal{H}}{ \rightleftharpoons}}
\providecommand{\system}{\overset{\mathcal{H}}{ \longleftrightarrow}}
\providecommand{\gauss}[2]{\mathcal{N}\ensuremath{\left(#1,#2\right)}}
	%\newcommand{\solution}[2]{\textbf{Solution:}{#1}}
\newcommand{\solution}{\noindent \textbf{Solution: }}
\providecommand{\dec}[2]{\ensuremath{\overset{#1}{\underset{#2}{\gtrless}}}}
%\numberwithin{equation}{section}
%\numberwithin{problem}{section}

\def\putbox#1#2#3{\makebox[0in][l]{\makebox[#1][l]{}\raisebox{\baselineskip}[0in][0in]{\raisebox{#2}[0in][0in]{#3}}}}
     \def\rightbox#1{\makebox[0in][r]{#1}}
     \def\centbox#1{\makebox[0in]{#1}}
     \def\topbox#1{\raisebox{-\baselineskip}[0in][0in]{#1}}
     \def\midbox#1{\raisebox{-0.5\baselineskip}[0in][0in]{#1}}


% paper title
% can use linebreaks \\ within to get better formatting as desired
\title{
\logo{
Mathematics through OPAMP 
}
}
%
%
% author names and IEEE memberships
% note positions of commas and nonbreaking spaces ( ~ ) LaTeX will not break
% a structure at a ~ so this keeps an author's name from being broken across
% two lines.
% use \thanks{} to gain access to the first footnote area
% a separate \thanks must be used for each paragraph as LaTeX2e's \thanks
% was not built to handle multiple paragraphs
%

%\author{Y Aditya, A Rathnakar and G V V Sharma$^{*}$% <-this % stops a space
\author{G V S S Praneeth Varma, CH. Gajendranath Choudhury and G V V Sharma %<-this  stops a space
\thanks{The authors are with the Department
of Electrical Engineering, IIT, Hyderabad
502285 India e-mail: \{gadepall\}@iith.ac.in. All material in the manuscript is released under GNU GPL.  Free to use for all.
}}



% make the title area
\maketitle

\tableofcontents

\bigskip

\begin{abstract}
%\boldmath
This manual shows how to use an OPAMP for implementing mathematical functions.

\end{abstract}
%\begin{problem}
%Design a circuit for
%\begin{equation}
%y = kx, \quad k > 0
%\end{equation}
%\end{problem}
%\solution  The desired circuit is available in Fig. \ref{fig:noninvert}.  Using node analysis at $v$ and the fact that $v=x$,
%\begin{align}
%\label{eq:noninvert_node}
%\frac{v }{R_i} + \frac{v-y}{R_f}  &= 0,
%\\
%v & = x
%\end{align}
%Thus, \eqref{eq:noninvert_node} qyields
%\begin{align}
%y = v\brak{1+\frac{R_f}{R_i}}
%\end{align}
%\begin{figure}[!h]
%\centering
%\resizebox {\columnwidth} {!} {
%\input{./figs/noninvert.tex}
%}
%\caption{$y = kx$.}
%\label{fig:noninvert}
%\end{figure}
\section{Linear Combination}
\begin{problem}
\label{prob:lincomb_node}
In Fig. \ref{fig:lincomb}, the current entering the $+$ and $-$ terminals of the opamp is 0. The voltages at both terminals is $v$. Show that
\begin{equation}
\label{eq:lincomb_node}
y = k_1x_1 - k_2x_2, \quad k_1, k_2 > 0
\end{equation}
where $x_1,x_2$ are the inputs.  Find the values of $k_1$ and $k_2$.
\end{problem}
%
\solution
Using node analysis,
\begin{align}
\frac{v-x_2}{R_2} + \frac{v-y}{R_f} &= 0
\\
\frac{v-x_1}{R_1} + \frac{v}{R_3} &= 0
\end{align}
resulting in
\begin{align}
v \brak{\frac{1}{R_2}+\frac{1}{R_f}} &= \frac{x_2}{R_2}+\frac{y}{R_f} 
\\
\frac{x_1}{R_1} &= v\brak{\frac{1}{R_1}+\frac{1}{R_3}}
\end{align}
Simplifying,
\begin{align}
\frac{x_1}{R_1}\brak{\frac{1}{R_2}+\frac{1}{R_f}} &= \frac{x_2}{R_2}\brak{\frac{1}{R_1}+\frac{1}{R_3}} + \frac{y}{R_f} \brak{\frac{1}{R_1}+\frac{1}{R_3}}
\\
\implies y &= \frac{ -\frac{x_2}{R_2}\brak{\frac{1}{R_1}+\frac{1}{R_3}} +\frac{x_1}{R_1}\brak{\frac{1}{R_2}+\frac{1}{R_f}} }{\frac{1}{R_f}\brak{\frac{1}{R_1}+\frac{1}{R_3}}}
\\
&=  \underbrace{\frac{R_3\brak{{R_2}+{R_f}}}{R_1\brak{{R_1}+{R_3}}}}_{k_1}x_1-\underbrace{\frac{R_f}{R_2}}_{k_2}x_2
\\
&= k_1 x_1 - k_2 x_2
\end{align}

\begin{figure}[!h]
\centering
\resizebox {\columnwidth} {!} {
\begin{circuitikz}
  \draw (0,0) node[op amp] (opamp) {}
  (opamp.-) to [R, l_=$R_2$, *-o] ($(opamp.-)-(2,0)$) node[left]{$x_{2}$}
  (opamp.-) |- ($(opamp.-)+(0.2,1)$) to[R=$R_f$] ($(opamp.-)+(2.2,1)$) -|
  (opamp.out) to[short,*-] ($(opamp.out)+(.5,0)$) node [right] {$y$} node [ocirc] {} 
  (opamp.+) to [R, l_=$R_1$, *-o] ($(opamp.+)-(2,0)$) node[left]{$x_1$}
  (opamp.+) to [R, l_=$R_3$] ($(opamp.+)-(0,2)$) node[ground]{}  
  (opamp.-) node[below] {$v$}         
  (opamp.+) node[above] {$v$}           
  ;
\end{circuitikz}


}
\caption{$y = k_1x_1 - k_2x_2$.}
\label{fig:lincomb}
\end{figure}
\begin{problem}
Design a circuit for 
\begin{equation}
\label{eq:lin_pos}
y = kx, \quad k > 0
\end{equation}
\end{problem}
\begin{problem}
Design a circuit for 
\begin{equation}
\label{eq:lin_neg}
y = -kx, \quad k > 0
\end{equation}
\end{problem}
\section{Differentiation and Integration}
\begin{problem}
\label{prob:derivative}
Design a circuit for 
\begin{equation}
x = -k\frac{dy}{dt},  k > 0
\end{equation}
and obtain an expression for $k$.
\end{problem}
\solution
Fig. \ref{fig:derivative} provides the solution which is explained below. Using node analysis in the $s$ domain,
\begin{align}
\frac{V(s)- Y(s)}{R} + sC{V(s)-X(s)} = 0
\end{align}
%
Since $v = 0$, the above equation results in
\begin{align}
Y(s) &= -sCR X(s)
\\
\implies y(t) &= -\underbrace{RC}_{k} \frac{dx}{dt}
\end{align}
\begin{figure}[!h]
\centering
\resizebox {\columnwidth} {!} {
%\documentclass{standalone}
%\usepackage{circuitikz}
%
%\begin{document}
%
%\begin{circuitikz}
%  [
%    point/.style = {draw, circle,  fill = black, inner sep = 0.5pt},
%  ]
\begin{circuitikz}
  \draw (0,0) node[op amp] (opamp) {}
  (opamp.-) to[short,*-]  ($(opamp.-)+(0,.5)$) node[ground,rotate=180] {}  
  (opamp.+) |- ($(opamp.+)+(0.2,-1)$) to[R=$R$] ($(opamp.+)+(2.2,-1)$) -|
  (opamp.out) to[short,*-] ($(opamp.out)+(.5,0)$) node [right] {$y$} node [ocirc] {} 
  (opamp.+) to [C, l_=$C$, *-o] ($(opamp.+)-(2,0)$) node[left]{$x$}
  (opamp.-) node[below] {$v$}         
  (opamp.+) node[above] {$v$}           
  ;
\end{circuitikz}
%\end{document}

}
\caption{$y = -k\frac{dx}{dt}$.}
\label{fig:derivative}
\end{figure}
%
\begin{problem}
Modify the circuit in Problem \ref{prob:derivative} to obtain
%
\begin{equation}
y = k\int x(t)\,dt
\end{equation}
%
Is $k > 0?$.
\end{problem}
\begin{problem}
How will you obtain 
%
\begin{equation}
y = k \frac{dx}{dt}, \quad k > 0 ?
\end{equation}
%
\end{problem}
\section{Practical Verification}
In the following, $1k\Omega$ and $2k\Omega$ resistances are available.
\begin{problem}
Verify your circuit in Problem \ref{prob:lincomb_node} for $x_1 = 0.5 V, x_2 = 1 V, y = -1.1. V $.
\end{problem}
\begin{problem}
Verify  your circuit for \eqref{eq:lin_pos} for  $x = 0.5 V$ and $y = 1.5 V$. You will have to choose the resistances appropriately.
\end{problem}
\begin{problem}
Verify your circuit for \eqref{eq:lin_neg} if $x = 1 V, y = -2 V$.
\end{problem}
\begin{problem}
A triangular wave with $V_{pp} = 1V$ and frequency  100 Hz is given as the input in Problem \ref{prob:derivative}.  What is the output for $C = 1 \mu F$ and $R = 10 k\Omega$? Verify your
result.
\end{problem}


\end{document}


