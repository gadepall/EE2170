\documentclass{standalone}
\usepackage{circuitikz}

\begin{document}

\begin{circuitikz}
  [
    point/.style = {draw, circle,  fill = black, inner sep = 0.5pt},
  ]
%\begin{circuitikz}
  \draw (0,0) node[op amp] (opamp) {}
  (opamp.-) to [R, l_=$R_1$, *-o] ($(opamp.-)-(2,0)$) node[left]{$x_{1}$}
  (opamp.-) |- ($(opamp.-)+(0.2,1)$) to[R=$R_f$] ($(opamp.-)+(2.2,1)$) -|
  (opamp.out) to[short,*-] ($(opamp.out)+(.5,0)$) node [right] {$y$} node [ocirc] {} 
  (opamp.+) to [R, l_=$R_2$, *-o] ($(opamp.+)-(2,0)$) node[left]{$x_2$}
  (opamp.+) to [R, l_=$R_3$] ($(opamp.+)-(0,2)$) node[ground]{}  
  (opamp.-) node[below] {$v$}         
  (opamp.+) node[above] {$v$}           
%  (opamp.+) to[short]  ($(opamp.+)-(0,.5)$) node[ground] {}  
  ;
\end{circuitikz}
%\node (V_o) at (2,0) [point,label = right:$y$] {};
%\node (v) at (-1.2,-2.0) [point,label = above left:$v$] {};
%\node (V_i) at (-3,0.5) [point,label = left:$x_2$] {};
%\node (x_2) at (-3,-2) [point,label = left:$x_1$] {};
%  \draw
%  (0, 0) node[op amp] (opamp) {}
%      (opamp.-) to[R=$R_2$] (V_i)
%(opamp.-) node[below] {$v$}        
%  (opamp.-) to[short,*-] ++(0,1.5) coordinate (leftC)
%  to[R=$R_f$] (leftC -| opamp.out)
%  to[short,-*] (opamp.out)      
%%  (opamp.+) to[short,*-] ++(0,-1.5)  coordinate (leftC)
%
%  (opamp.out) node[right] {}
%   (opamp.out) -- (V_o)
%%  (leftC) to [R=$R_3$] ++(-0,-2.5) coordinate (gnd)
%  (opamp.+) to [R=$R_3$]  to node[ground] {}
% (opamp.+) to[R=$R_1$] (x_2)
%;\end{circuitikz}
\end{document}
